%----------------------------------------------------------------------
%	SUMMARY CHAPTER
%----------------------------------------------------------------------
\label{ch:summary}

\section{Results}
A DIRC detector is ideal for meeting the hadronic PID requirements in the barrel region of an EIC due to its small radial footprint and excellent particle separation capabilities at sub-10~GeV/c particle momentum. The current baseline design of the EIC DIRC is based on the compact PANDA DIRC design \cite{PANDA_barrel}, featuring a compact expansion volume and lens-based focusing. Both geometric and time-based reconstruction analysis were done on the GEANT4 simulation of the EIC DIRC. For the geometric reconstruction Figure \ref{fig:EIC_track_res} shows a reasonable performance that would allow for $3\sigma$ $\pi/K$ separation (see Figure \ref{fig:PID_performance} for reference) at 6~GeV/c momentum for most polar angles and dropping only slightly below $3\sigma$ if it is assumed that a large correlated term of 1~mrad will be seen in the actual experiment. In the case of the time-based reconstruction (Figure \ref{fig:EIC_timebased_performance}) the simulation again predicts a $3\sigma$ separation for a majority of the polar angle range of the detector, with performance dropping for polar angles near perpendicular.

While it would be ideal to test these simulation results directly with physical measurements, at this stage of the R\&D effort it is more sensible to take advantage of the synergy between the EIC DIRC group and other DIRC groups rather than spend some large fraction of the PID R\&D budget on a single EIC DIRC prototype. A synergistic test beam campaign was carried out during the summers of 2015 and 2016 with the PANDA Barrel DIRC group to study the performance of a PANDA DIRC detector prototype using the components envisioned for the EIC DIRC, namely a new 3-layer lens focusing optic designed to have a flat focal plane across the face of the MCP-PMT detector plane. Verification that the GEANT4 simulation using this PANDA prototype geometry agrees with experimental data is key in ensuring that the predicted performance of the EIC DIRC is valid.

Along with the analysis of the performance of the EIC DIRC 3-layer lens in a particle beam, it was also necessary to investigate the radiation hardness of the center layer of lanthanum crown glass \cite{SchottData}, NLaK33, as well as the actual shape of the focal plane to compare with simulation. Measurements of the radiation hardness were carried out at the Catholic University of America using  160~keV X-ray cabinet \cite{XRayCabinet} for irradiation and a monochromator \cite{Monochromator} for measuring the transmission of the glass after each irradiation step. It was found that the glass suffers approximately 1.3\% transmission loss per 100~rad of delivered dose (Figure \ref{fig:transmission_measurements}). It is, at the time of this writing, unknown what the expected dose delivered to the DIRC detector at an EIC will be over the lifetime of the experiment. Alternatives and solutions are discussed in the next section.

Measurements of the focal plane of the 3-layer lens were done at Old Dominion University using a custom-built laser setup and 3D printed lens holder. Initial measurements showed a systematic shift in the position of the focal plane between data and simulation by roughly 4~cm. After many measurements and adjustments to the setup it was found that the cause of this shift was most likely due to a non-zero angle between the two laser beams of roughly 0.15~mrad. After this adjustment was implemented in the simulation, the measured data very nicely reproduces both the shape and position of the predicted focal plane for multiple tilt angles and even when shifting the beam off-center of the lens.

The analysis of the 2015 CERN test beam data from the PANDA DIRC prototype focused primarily on the configuration with the 3-layer lens, bar radiator, and 7 GeV/c beam momentum in an attempt to closely match the parameters of the EIC DIRC baseline design. Both the geometric and time-based reconstruction methods were used to determine the performance of the prototype. The GEANT4 simulation is in good agreement with the results of the analyzed experimental data, which gives confidence to the results presented in Chapter \ref{ch:eicdirc} for the EIC DIRC that the desired PID performance can be achieved by such a detector. 

Confirmation that the GEANT4 simulation package for the EIC DIRC agrees with physical data is crucial in proceeding with the R\&D effort. A synergistic test beam campaign was carried out with the PANDA Barrel DIRC group from GSI at CERN in the summer of 2015 using a prototype detector modeled after the baseline design of the PANDA DIRC. During the 2015 CERN test beam campaign multiple configurations of beam momentum, radiators, and focusing optics were tested (see Table \ref{tab:runs2015}). Study 151, using the 3-layer lens, a bar radiator, and 7 GeV/c beam momentum, was used in the main analysis for this thesis due to its similarities to the baseline design of the EIC DIRC. Both the geometric and time-based reconstruction methods were used to determine the performance of the prototype. The GEANT4 simulation is in good agreement with the results of the analyzed experimental data, which gives confidence to the results presented in Chapter \ref{ch:eicdirc} for the EIC DIRC that the desired PID performance can be achieved by such a detector. 

\section{Future Work}
There are still several steps to take in the R\&D effort for the EIC DIRC: further studies of the radiation hardness of NLaK33, alternative materials for the lens design, and building a full, baseline-design-compatible EIC DIRC prototype.

Tests of the radiation hardness of NLaK33 were done with a somewhat thick (1~cm) piece of glass. The central layer of the 3-layer lens design, however, is set to 0.56~cm at the thickest portion, and thins out to 0.2~cm at the edge (see Figure \ref{fig:3CS_schematic}). It is unclear what the penetration depth of NLaK33 is, and therefore how big of a change a smaller volume of material would have on the transmission. Talks are currently underway with a manufacturer to procure a piece of glass with a smaller thickness to test the penetration depth.

Along with testing the penetration depth of NLaK33, tests are also planned for exposing the material to neutron radiation. Again, the neutron flux at the DIRC detector in an EIC is unclear, but having a feel for the type of neutron damage the material can withstand will help in the development of the lens.

If, after all radiation tests of the thinner NLaK33 piece are complete, it is found that it will lose as significant amount of transmission after a relatively short time of running then an alternative material must be found for the lens. Currently there are investigations into making the lens out of a different material called Lead fluoride (PbF$_2$). PbF$_2$ is ideal because of its high refractive index, similar to that of NLaK33, and its proven high radiation hardness \cite{PbF2RadHard}. The challenge with using PbF$_2$ in the lens is that many manufacturers are unwilling to work with it due to the fear of contamination of their tools with lead.

In order to fully test the EIC DIRC design a prototype must be constructed and tested in a hadron beam. To carry out such a test beam campaign, MCP-PMTs with appropriately sized $3\times3\unit{mm}^2$ pixels (crucial for the desired resolution) along with a correctly sized expansion volume and radiator bars must be procured. Costs can be somewhat mitigated if radiator bars from previous experiments could be used instead of purchasing new bars. It is currently planned to include the costs of a full test beam in the US as part of the 2019 EIC budget for detector R\&D.

In conclusion, a DIRC detector is an ideal solution for hardronic PID in the barrel region around the electron/ion interaction point of an EIC due to its compact radial size and resolving power for charged particles with sub-10~GeV/c momentum. Many milestones have so far been achieved in the R\&D efforts, including the verification of the EIC DIRC simulation package via the 2015 CERN test beam, confirmation of the shape of the new 3-layer spherical lens design at ODU, limited radiation hardness testing of the NLaK33 material at CUA, and extensive studies of the influence of high magnetic fields on MCP-PMTs at JLab. 