%----------------------------------------------------------------------
%	INTRODUCTORY CHAPTER
%----------------------------------------------------------------------
\label{ch:intro}
The Electron-Ion Collider (EIC) is planned as the Department of Energy's next big nuclear physics facility to be built in the Unite States. It will be the world's first collider with polarized electron and ion beams, as well as having the capability of delivering beams of heavier, unpolarized ions. Electron beam energies reaching up to $2-21$~GeV/c and proton beam energies up to $250$~GeV/c necessitate a sophisticated $4\pi$ detector. 

Excellent hadronic particle identification (PID) not only at the end-caps but also in the barrel region around the beam-beam interaction point is crucial for the success of the physics program of an EIC. For the end-caps there is ample space for Ring Imaging Cherenkov (RICH) detectors which have been shown to provide excellent PID for large momentum particles, however, due to the limited space available in the barrel region a different approach must be taken. A modified RICH detector, known as a DIRC (Detection of Internally Reflected Cherenkov light) is an attractive solution for PID of particles with large transverse momentum transfer as it occupies less than 5~cm of radial space while still providing excellent PID performance, as shown by the performance of the BaBar barrel DIRC \cite{BaBarDIRC}.

Although based on the design of BaBar's barrel DIRC, a DIRC at an EIC presented many challenges in reaching the required $\pi/K$ separation power of $3\sigma$ at 6~GeV/c particle momentum. A more compact expansion volume necessitated the design of a new 3-layer spherical lens focusing optic to improve resolution. Testing of this new lens installed in a prototype DIRC detector was done in a particle beam at the CERN Proton Synchrotron (PS) to test and compare the performance of the 3-layer lens design with other focusing and radiator options.

Along with performance in a particle beam the new lens was also subjected to radiation hardness tests using a 160~keV X-ray source to determine the durability of the lanthanum crown glass used for the middle layer of the lens. The "flat" focal plane of a prototype lens was also measured and compared to simulation prediction.